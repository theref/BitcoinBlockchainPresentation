\documentclass{beamer}

\usepackage{booktabs}
\usepackage{graphicx}
\usepackage{minted}
\usemintedstyle[python]{solarizedlight}
\usepackage{tikz}
\usetikzlibrary{shapes, arrows}
\usetheme{metropolis}
\usepackage[backend=biber]{biblatex}
\bibliography{BitcoinBlockchain}


\title{An introduction to Bitcoin and Blockchains}
\author{James Campbell}
\institute[Cardiff University]
  {
  Department of Mathematics\\
  Cardiff University
  }

\begin{document}

\begin{frame}
  \titlepage
\end{frame}


\section{History and Background}


\begin{frame}{Bitcoin: The first Blockchain}
\begin{columns}
  \begin{column}{0.5\textwidth}
    \begin{itemize}
      \item Outlined in October 2008 by Satoshi Nakamoto \cite{Nakamoto}
      \item Network online January 2009 \cite{GenesisBlock}
      \item First transaction for physical goods in May 2010
    \end{itemize}
  \end{column}
  \begin{column}{0.5\textwidth} 
    \begin{center}
      \includegraphics[width=0.9\textwidth]{images/the_times.jpg}
    \end{center}
  \end{column}
\end{columns}
\end{frame}


\begin{frame}{Definition}
\emph{A Transparent, Public, Distributed, Append-Only Ledger}\\ - Unknown
\end{frame}


\begin{frame}{The Ledger}
\begin{center}
  \begin{tabular}{llr} \toprule
    From & To & Amount (£)\\
    \midrule
    & \dots & \\
    Amy & Ben & 13.65 \\
    Ben & Tesco & 1.01 \\
    Lidl & Amy & 492.50 \\
    & \dots & \\
  \end{tabular}
\end{center}
\end{frame}


\section{Tell Me More...}


\begin{frame}{Hash Function}
\emph{A hash function is any function that can be used to \alert{securely} map data of arbitrary size to data of fixed size.}\\
\\
Cryptographic Hash Functions have some useful properties:
\begin{itemize}
  \item Deterministic and Fast
  \item Non-Invertible
  \item Collision Resistant
  \item Avalanche Effect
\end{itemize}
\end{frame}


\begin{frame}{A Block}
\begin{itemize}
  \item An Index
  \item Some Data
  \item Nonce
  \item Hashable
\end{itemize}
\end{frame}


\begin{frame}{Mining a Block}
A Block is not \alert{valid} unless it's Hash satisfies a certain criteria.
\end{frame}


\begin{frame}{A Blockchain}
Link Blocks together (in a chain) by including the Hash of the previous Block.\\
\\
\pause
What happens if an attacker attempts to edit some previous data?
\end{frame}


\section{Tell Me More...}


\begin{frame}{The Blockchain Process}
\begin{enumerate}
  \item A user creates an addition to the ledger \pause
  \item This is broadcast to the entire network \pause
  \item Nodes in the network attempt to create a Block \pause
  \item The new Block is added to the Blockchain and broadcast to the Network \pause
  \item The process repeats \pause
\end{enumerate}

\alert{Nodes in the network use the longest Blockchain available}
\end{frame}


\begin{frame}{A Malicious Node}
\tikzstyle{block} = [draw, fill=blue!20, rectangle, minimum height=2em, minimum width=2em]
\tikzstyle{bblock} = [draw, fill=red!20, rectangle, minimum height=2em, minimum width=2em]

\begin{center}
  \begin{tikzpicture}[auto, node distance=1.5cm,>=latex']
    \node [coordinate] (start) {};
    \node [block, right of=start] (block1) {$B_1$};
    \node [block, right of=block1] (block2) {$B_2$};
    \node [block, right of=block2] (block3) {$B_3$};
    \node [block, right of=block3] (block4) {$B_4$};
    \node [block, right of=block4] (block5) {$B_5$};
    \node [coordinate, right of=block5] (out) {};

    \node [coordinate, below of=block2] (fork) {};

    \node [bblock, below of=block3] (block3m) {$\hat{B_3}$};
    \node [bblock, below of=block4] (block4m) {$\hat{B_4}$};
    \node [bblock, below of=block5] (block5m) {$\hat{B_5}$};
    \node [coordinate, below of=out] (outm) {};

    \draw [->] (start) -- (block1);
    \draw [->] (block1) -- (block2);
    \draw [->] (block2) -- (block3);
    \draw [->] (block3) -- (block4);
    \draw [->] (block4) -- (block5);
    \draw [->] (block5) -- (out);

    \draw [->] (block2) -- (fork) -- (block3m);
    \draw [->] (block3m) -- (block4m);
    \draw [->] (block4m) -- (block5m);
    \draw [->] (block5m) -- (outm);
  \end{tikzpicture}
\end{center}
\pause
\alert{The network is secure provided that no single malicious attacker controls more than half the hashing power.}
\end{frame}


\section{Tell Me More About Bitcoin...}


\begin{frame}{Unanswered Issues}
\begin{itemize}
  \item What incentive is there for Good nodes?
  \item How are Bitcoins created?
  \item Proof of Ownership
\end{itemize}
\end{frame}


\begin{frame}{Coinbase Transaction}
Every Block includes a transaction that has \alert{no sender}, only a recipient.\pause
\\
\\
The recipient is the Node that found the correct Nonce to validate the Block.\pause
\\
\\
Called the Block Reward (which decreases over time).
\end{frame}


\begin{frame}{Public/Private Key Pairs}
Bitcoin addresses are \alert{Public Keys}.\\
\\ \pause
Every transaction must be signed by the \alert{Private Key} that is paired with the senders \alert{Public Key}.\\
\\ \pause
Anyone can easily verify that a transaction was created by the sender \textbf{only}.
\end{frame}


\section{Tell Me More...}


\begin{frame}{Further Reading (Bitcoin Related)}
\begin{itemize}
  \item Myths \cite{Kikvadze_2018,Top_10_Myths_About_Bitcoin_2016}
  \item Environmental Impact of Bitcoin \cite{Buntix,Ou_2017,steembusiness_2017,Kobie}
  \item Merkle Trees \cite{Ray_2017,Merkle}
  \item Segwit and Lightning Network \cite{Investopedia_2017,Lee_2017,Wirdum,Waee_Digital_Ltd}
\end{itemize}
\end{frame}


\begin{frame}{Further Reading (Future of Blockchain)}
\begin{itemize}
  \item Proof of Stake \cite{Investopedia_2017b,Rosic_2017}
  \item Smart Contracts \cite{frozeman_vbuterin_chriseth_debris_2018,K_2017}
  \item DAG currencies (IOTA, Nano, ByteBall) \cite{Ryszkiewicz,jimmco_2017,Popov_2017}
  \item Hashgraph \cite{Baird_2016,Samman}

\end{itemize}
\end{frame}

\section{Thank you...any questions?}


\begin{frame}[allowframebreaks]
  \printbibliography
\end{frame}

\end{document}